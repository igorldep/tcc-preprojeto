%% Baseado no arquivo: 
%% abtex2-modelo-trabalho-academico.tex, v-1.9.6 laurocesar
%% by abnTeX2 group at http://www.abntex.net.br/ 
%% Adaptado para um modelo dssse TCC (Graduação)

% ------------------------------------------------------------------------
% ------------------------------------------------------------------------
% abnTeX2: Modelo de Trabalho Academico (tese de doutorado, dissertacao de
% mestrado e trabalhos monograficos em geral) em conformidade com 
% ABNT NBR 14724:2011: Informacao e documentacao - Trabalhos academicos -
% Apresentacao
% ------------------------------------------------------------------------
% ------------------------------------------------------------------------

\documentclass[
	% -- opções da classe memoir --
	12pt,				% tamanho da fonte
	openright,			% capítulos começam em pág ímpar (insere página vazia caso preciso)
	twoside,			% para impressão em recto e verso. Oposto a oneside
	a4paper,			% tamanho do papel. 
	% -- opções da classe abntex2 --
	%chapter=TITLE,		% títulos de capítulos convertidos em letras maiúsculas
	%section=TITLE,		% títulos de seções convertidos em letras maiúsculas
	%subsection=TITLE,	% títulos de subseções convertidos em letras maiúsculas
	%subsubsection=TITLE,% títulos de subsubseções convertidos em letras maiúsculas
	% -- opções do pacote babel --
	english,			% idioma adicional para hifenização
	%french,				% idioma adicional para hifenização
	%spanish,			% idioma adicional para hifenização
	brazil				% o último idioma é o principal do documento
	]{abntex2}

% ---
% Pacotes básicos 
% ---
\usepackage{lmodern}			% Usa a fonte Latin Modern			
\usepackage[T1]{fontenc}		% Selecao de codigos de fonte.
\usepackage[utf8]{inputenc}		% Codificacao do documento (conversão automática dos acentos)
\titulo{Segurança da Internet das Coisas com API utilizando \textit{BlockChain}}
\usepackage{indentfirst}		% Indenta o primeiro parágrafo de cada seção.
\usepackage{color}				% Controle das cores
\usepackage{graphicx}			% Inclusão de gráficos
\usepackage{microtype} 			% para melhorias de justificação
% ---
		

% ---
% Pacotes de citações
% ---
\usepackage[brazilian,hyperpageref]{backref}	 % Paginas com as citações na bibl
\usepackage[alf]{abntex2cite}	% Citações padrão ABNT

% --- 
% CONFIGURAÇÕES DE PACOTES
% --- 

% ---
% Configurações do pacote backref
% Usado sem a opção hyperpageref de backref
\renewcommand{\backrefpagesname}{%Citado na(s) página(s):~
}
% Texto padrão antes do número das páginas
\renewcommand{\backref}{}
% Define os textos da citação
\renewcommand*{\backrefalt}[4]{
	%\ifcase #1 %
	%	Nenhuma citação no texto.%
	%\or
	%	Citado na página #2.%
	%\else
	%	Citado #1 vezes nas páginas #2.%
	%\fi
    }%
% ---

% ---
% Informações de dados para CAPA e FOLHA DE ROSTO
% ---
\titulo{Abordagens para a migração de aplicações \textit{On-Premise} para \textit{Cloud Computing}}
\autor{Igor Luciano de Paula}
\local{Belo Horizonte}
\data{2022}
\orientador{Andrei Rimsa}

\instituicao{%
  Centro Federal de Educação Tecnológica de Minas Gerais -- CEFET-MG
  \par
  Departamento de Computação
  \par
  Curso de Engenharia da Computação
  }
\tipotrabalho{Monografia (Graduação)}
% O preambulo deve conter o tipo do trabalho, o objetivo, 
% o nome da instituição e a área de concentração 
\preambulo{Trabalho de Conclusão de Curso apresentado ao Curso
de Engenharia de Computação do Centro Federal de
Educação Tecnológica de Minas Gerais, como requisito
parcial para a obtenção do título de Bacharel em
Engenharia de Computação.}
% ---


% ---
% Configurações de aparência do PDF final

% alterando o aspecto da cor azul
\definecolor{blue}{RGB}{41,5,195}

% informações do PDF
\makeatletter
\hypersetup{
     	%pagebackref=true,
		pdftitle={\@title}, 
		pdfauthor={\@author},
    	pdfsubject={\imprimirpreambulo},
	    pdfcreator={LaTeX with abnTeX2},
		pdfkeywords={abnt}{latex}{abntex}{abntex2}{trabalho acadêmico}, 
		colorlinks=true,       		% false: boxed links; true: colored links
    	linkcolor=blue,          	% color of internal links
    	citecolor=blue,        		% color of links to bibliography
    	filecolor=magenta,      		% color of file links
		urlcolor=blue,
		bookmarksdepth=4
}
\makeatother
% --- 

% --- 
% Espaçamentos entre linhas e parágrafos 
% --- 

% O tamanho do parágrafo é dado por:
\setlength{\parindent}{1.3cm}

% Controle do espaçamento entre um parágrafo e outro:
\setlength{\parskip}{0.2cm}  % tente também \onelineskip

% ---
% compila o indice
% ---
\makeindex
% ---

% ----
% Início do documento
% ----
\begin{document}

% Seleciona o idioma do documento (conforme pacotes do babel)
%\selectlanguage{english}
%\selectlanguage{brazil}

% Retira espaço extra obsoleto entre as frases.
\frenchspacing 

% ----------------------------------------------------------
% ELEMENTOS PRÉ-TEXTUAIS
% ----------------------------------------------------------



%% Baseado no arquivo: 
%% abtex2-modelo-trabalho-academico.tex, v-1.9.6 laurocesar
%% by abnTeX2 group at http://www.abntex.net.br/ 
%% Adaptado para um modelo de TCC (Graduação)

% ---
% Capa
% ---
\imprimircapa
% ---

% ---
% Folha de rosto
% (o * indica que haverá a ficha bibliográfica)
% ---
\imprimirfolhaderosto*

% ---
%\pdfbookmark[0]{\contentsname}{toc}
%\tableofcontents*
\cleardoublepage
% ---

% ----------------------------------------------------------
% ELEMENTOS TEXTUAIS
% ----------------------------------------------------------
\textual

% ----------------------------------------------------------
% Como o documento será grande, sugiro dividir em diversos arquivos, um para cada capítulo.
% ----------------------------------------------------------


\chapter[Introdução]{Introdução}
\label{cap:introducao}

Com o crescimento do mercado global de Tecnologia da Informação (TI) cresce também nossa dependência dos serviços de TI. Cloud Computing é uma abordagem para o compartilhamento de recursos que tem profundas consequências na forma como o mundo faz negócios e como
interagimos uns com os outros, e o impacto total ainda não sabemos. Vivemos em uma sociedade cada vez mais dependente da tecnologia e, à medida que aumentamos nossa presença na esfera
virtual, os nossos dados tornam-se dispersos em meios que têm pouco controle \cite{veras2012}.

A oferta do software On-Premise é um modelo de distribuição de software amplamente utilizado na indústria. Para este trabalho, iremos considerar On-Premise como uma distribuição onde o sistema é instalado localmente no usuário final ou em uma servidor privado e interno(no fornecedor), onde o usuário consome o serviço. A nuvem é um novo paradígma para realizar essa distribuição.

%(\textit{Internet of Things} - IoT) emergiu como uma área de impacto, potencial, e crescimento, com a Cisco Inc. prevendo a existência de 50 bilhões de dispositivos conectados até o final de 2020 \cite{khan2018iot}.

Agora, existem muitas definições e metáforas de computação em nuvem. Do nosso ponto de vista, a computação em nuvem é um tipo de técnica de computação em que os serviços de TI são fornecidos por grandes unidades de computação de baixo custo conectadas por redes IP. A computação em nuvem está enraizada no design da plataforma do mecanismo de pesquisa. Existem 5 grandes características técnicas da computação em nuvem: (1) recursos de computação em larga escala; (2) alta escalabilidade e elasticidade; (3) pool de recursos compartilhados (recurso virtual e físico); (4) agendamento dinâmico de recursos e; (5) propósito geral
\cite{qian2009}.

A computação em nuvem é um modelo para permitir acesso de rede onipresente, conveniente e sob demanda a um conjunto compartilhado de recursos de computação configuráveis (por exemplo, redes, servidores, armazenamento, aplicativos e serviços) que podem ser rapidamente provisionados e liberados com esforço mínimo de gerenciamento ou interação do provedor de serviços \cite{WPcloud}.

A nuvem oferece flexibilidade através de um provisionamento de serviços escalável, aproveitando os avanços da conectividade e as tecnologias de virtualização e muda a forma de fazer negócios. Ela vai transformar a forma de pensar os Datacenters e a concorrência entre operadores irá provavelmente forçar uma consolidação do mercado, deixando alguns operadores com um pequeno número de grandes instalações. A necessidade de minimizar o custo total de propriedade (Total Cost of Ownership – TCO) e a pressão sobre recursos continuarão a impulsionar a tendência para densidades de carga crescente, manutenção da eficiência energética e a priorização da alta disponibilidade para os operadores
\cite{veras2012}.


\newpage

% -- Escrever objetivo do trabalho

\section{Motivação e objetivos}

A solução On-Premise exige que o comprador da licença seja o total encarregado pela gestão da infraestrutura necessária para a aquisição, enquanto que na solução em forma de Software As A Service esse gerenciamento é feito pelo fornecedor dos serviços (SILVA, 2012).
Embora o preço de uma licença On-Premise de um Enterprise Resource System (ERP) possa ser relativamente alto, os custos de implantação e manutenção são os problemas mais comuns quando se trata de aspectos financeiros (KLOS; KREBS, 2008). 
Empresas como Oracle e SAP calcularam que o preço de implementação, juntamente com alguns custos operacionais ocultos chegam a ser de três a sete vezes o preço de licenciamento. Nessa conjuntura, é difícil para as pequenas e médias empresas adquirirem um sistema ERP (SAP, 2008). 

As configurações de hospedagem de aplicativos de software variam de soluções no local a nuvens privadas e nuvens públicas\cite{defCloud2010}. A migração para a nuvem ou entre esses cenários geralmente híbridos ou entre ofertas é uma preocupação fundamental e a determinação e avaliação de possíveis processos de migração são importantes\cite{comparison2013}.

Diante dessa nova tecnologia e de seus inúmeros benefícios para diversos cenários, o ramo da Arquitetura de Software olha cada vez mais para esse novo paradígma. Isso torna necessário o estudo de como realizar de maneira adequada a migração de sistemas atualmente rodando On-Premise, para serem executados em Cloud.

O objetivo desse trabalho é apresentar aspectos modernos, tanto a nível de revisão bibliográfica, como a nível de novos panoramas, sobre as diversas abordagens para a migração de Software rodando On-Premise para a Cloud.


\section{Relevância}
    Este trabalho apresentará princípios para a migração de sistemas para a núvem, trazendo também novas perspectivas e novas problemáticas para o problema proposto.
    Em termos científicos, este trabalho motivará novas pesquisas com o intuito aplicar os entendimentos discutidos, e de novas definições para os problemas citados.


%\chapter{Fundamentação Teórica}
\label{cap:fundamentacao_teorica}



\chapter{Metodologia}
\label{cap:metodologia}

Este trabalho será feito seguindo as seguintes etapas:

 \begin{enumerate}
   \item Levantamento e analise de trabalhos realizados na área de \textit{Cloud Computing}, \textit{On-Premises Systems} e Migração de sistemas;
   \item Validar soluções propostas anteriormente;
   \item Avaliar funcionamento dos conceitos anteriores;
   \item Levantamento de pontos em abertos sobre o problema;
   \item Escrever e entregar o TCC 1;
   \item Elaboração de soluções para os problemas propostos;
   \item Elaboração de validações das soluções propostas;
   \item Gerar conclusões e resultados bem embasados;
   \item Escrever e entregar o TCC 2.
 \end{enumerate}
 
\section{Infraestrutura Necessária}

Para a realização deste trabalho serão necessários os seguintes itens:

 \begin{enumerate}
   \item Um computador com o sistema operacional GNU/Linux;
   \item Um servidor On-Premise para testes;
   \item Um servidor em Cloud para testes.
 \end{enumerate}

\section{Resultados Esperados}
    O resultado esperado deste trabalho é apresentar novas perspectivas para os problemas na migração de um sistema On-Premise para Cloud.


\section{Cronograma}

\begin{tabular}{ |p{2cm}|p{0.6cm}|p{0.6cm}|p{0.6cm}|p{0.6cm}|p{0.6cm}|p{0.6cm}|p{0.6cm}|p{0.6cm}|p{0.6cm}|}
\hline
\rowcolor{lightgray} \multicolumn{10}{|c|}{Cronograma 2020 - 2021} \\
\hline
  Atividade &  abr & mai & jun & jul & ago & set & out & nov & dez\\
\hline
  1 & \cellcolor[HTML]{AA0044} & & & & & & & & \\
  \hline
  2 & \cellcolor[HTML]{AA0044} & \cellcolor[HTML]{AA0044} & & & & & & & \\
  \hline
  3 & &\cellcolor[HTML]{AA0044}& & & & & & &  \\
  \hline
  4 & & \cellcolor[HTML]{AA0044} & \cellcolor[HTML]{AA0044}& & & & & & \\
  \hline
  5 & & &\cellcolor[HTML]{AA0044} & \cellcolor[HTML]{AA0044}& & & & & \\
  \hline
  6 & & & & \cellcolor[HTML]{AA0044}& \cellcolor[HTML]{AA0044}& \cellcolor[HTML]{AA0044}& & & \\
  \hline
  7 & & & & & & &\cellcolor[HTML]{AA0044} & & \\
  \hline
  8 & & & & & & &\cellcolor[HTML]{AA0044} &\cellcolor[HTML]{AA0044} & \\
  \hline
  9 & & & & & & & & \cellcolor[HTML]{AA0044}& \cellcolor[HTML]{AA0044}\\
\hline
\end{tabular}
%\include{src/2_formato}

% ----------------------------------------------------------

% ----------------------------------------------------------
% Finaliza a parte no bookmark do PDF
% para que se inicie o bookmark na raiz
% e adiciona espaço de parte no Sumário
% ----------------------------------------------------------
\phantompart

% ----------------------------------------------------------
% ELEMENTOS PÓS-TEXTUAIS
% ----------------------------------------------------------

%% Baseado no arquivo: 
%% abtex2-modelo-trabalho-academico.tex, v-1.9.6 laurocesar
%% by abnTeX2 group at http://www.abntex.net.br/ 
%% Adaptado para um modelo de TCC (Graduação)

\postextual
% ----------------------------------------------------------

% ----------------------------------------------------------
% Referências bibliográficas
% ----------------------------------------------------------
\bibliography{cefet_mg_decom_abntex2}


\end{document}
