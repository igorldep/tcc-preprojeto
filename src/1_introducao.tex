\chapter[Introdução]{Introdução}
\label{cap:introducao}

Com o crescimento do mercado global de Tecnologia da Informação (TI) cresce também nossa dependência dos serviços de TI. Cloud Computing é uma abordagem para o compartilhamento de recursos que tem profundas consequências na forma como o mundo faz negócios e como
interagimos uns com os outros, e o impacto total ainda não sabemos. Vivemos em uma sociedade cada vez mais dependente da tecnologia e, à medida que aumentamos nossa presença na esfera
virtual, os nossos dados tornam-se dispersos em meios que têm pouco controle \cite{veras2012}.

A oferta do software On-Premise é um modelo de distribuição de software amplamente utilizado na indústria. Para este trabalho, iremos considerar On-Premise como uma distribuição onde o sistema é instalado localmente no usuário final ou em uma servidor privado e interno(no fornecedor), onde o usuário consome o serviço. A nuvem é um novo paradígma para realizar essa distribuição.

%(\textit{Internet of Things} - IoT) emergiu como uma área de impacto, potencial, e crescimento, com a Cisco Inc. prevendo a existência de 50 bilhões de dispositivos conectados até o final de 2020 \cite{khan2018iot}.

Agora, existem muitas definições e metáforas de computação em nuvem. Do nosso ponto de vista, a computação em nuvem é um tipo de técnica de computação em que os serviços de TI são fornecidos por grandes unidades de computação de baixo custo conectadas por redes IP. A computação em nuvem está enraizada no design da plataforma do mecanismo de pesquisa. Existem 5 grandes características técnicas da computação em nuvem: (1) recursos de computação em larga escala; (2) alta escalabilidade e elasticidade; (3) pool de recursos compartilhados (recurso virtual e físico); (4) agendamento dinâmico de recursos e; (5) propósito geral
\cite{qian2009}.

A computação em nuvem é um modelo para permitir acesso de rede onipresente, conveniente e sob demanda a um conjunto compartilhado de recursos de computação configuráveis (por exemplo, redes, servidores, armazenamento, aplicativos e serviços) que podem ser rapidamente provisionados e liberados com esforço mínimo de gerenciamento ou interação do provedor de serviços \cite{WPcloud}.

A nuvem oferece flexibilidade através de um provisionamento de serviços escalável, aproveitando os avanços da conectividade e as tecnologias de virtualização e muda a forma de fazer negócios. Ela vai transformar a forma de pensar os Datacenters e a concorrência entre operadores irá provavelmente forçar uma consolidação do mercado, deixando alguns operadores com um pequeno número de grandes instalações. A necessidade de minimizar o custo total de propriedade (Total Cost of Ownership – TCO) e a pressão sobre recursos continuarão a impulsionar a tendência para densidades de carga crescente, manutenção da eficiência energética e a priorização da alta disponibilidade para os operadores
\cite{veras2012}.


\newpage

% -- Escrever objetivo do trabalho

\section{Motivação e objetivos}

A solução On-Premise exige que o comprador da licença seja o total encarregado pela gestão da infraestrutura necessária para a aquisição, enquanto que na solução em forma de Software As A Service esse gerenciamento é feito pelo fornecedor dos serviços (SILVA, 2012).
Embora o preço de uma licença On-Premise de um Enterprise Resource System (ERP) possa ser relativamente alto, os custos de implantação e manutenção são os problemas mais comuns quando se trata de aspectos financeiros (KLOS; KREBS, 2008). 
Empresas como Oracle e SAP calcularam que o preço de implementação, juntamente com alguns custos operacionais ocultos chegam a ser de três a sete vezes o preço de licenciamento. Nessa conjuntura, é difícil para as pequenas e médias empresas adquirirem um sistema ERP (SAP, 2008). 

As configurações de hospedagem de aplicativos de software variam de soluções no local a nuvens privadas e nuvens públicas\cite{defCloud2010}. A migração para a nuvem ou entre esses cenários geralmente híbridos ou entre ofertas é uma preocupação fundamental e a determinação e avaliação de possíveis processos de migração são importantes\cite{comparison2013}.

Diante dessa nova tecnologia e de seus inúmeros benefícios para diversos cenários, o ramo da Arquitetura de Software olha cada vez mais para esse novo paradígma. Isso torna necessário o estudo de como realizar de maneira adequada a migração de sistemas atualmente rodando On-Premise, para serem executados em Cloud.

O objetivo desse trabalho é apresentar aspectos modernos, tanto a nível de revisão bibliográfica, como a nível de novos panoramas, sobre as diversas abordagens para a migração de Software rodando On-Premise para a Cloud.


\section{Relevância}
    Este trabalho apresentará princípios para a migração de sistemas para a núvem, trazendo também novas perspectivas e novas problemáticas para o problema proposto.
    Em termos científicos, este trabalho motivará novas pesquisas com o intuito aplicar os entendimentos discutidos, e de novas definições para os problemas citados.

